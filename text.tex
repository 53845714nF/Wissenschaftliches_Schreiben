\section{Einleitung}
 Als eine moderne Programmiersprache hat Julia die Aufmerksamkeit von Entwicklern und Forschern auf der ganzen Welt erregt, dies beweisen nicht zuletzt über vierzigtausend Stars auf Github.
 \footnote{\url{https://github.com/JuliaLang/julia} Zugriff am 14.12.23} 
 
 Sie hat bei Anwendern auch den Ruf erworben, eine außergewöhnliche \glqq Sexiness\grqq{} in ihre Syntax und Funktionen zu integrieren. In dieser Arbeit werden wir die Faszination und Anziehungskraft der Julia-Programmiersprache erkunden. Eine Anziehung, die durch die elegante Verschmelzung von Geschwindigkeit und Vielseitigkeit entsteht. Im Fokus stehen hier die Geschichte und ihre heutige Bedeutung.

%Einleitung 5-10\% -> 50-100 Wörtern
% 79 Wörter 

\section{Geschichte}
\subsection{Entstehung von Julia}
Ausgangspunkt für die Entwicklung der Programmiersprache Julia war, dass viele aus der Wissenschaftsgemeinschaft und Datenanalysten auf verschiedenste Programmiersprachen zurückgrifffen, darunter Lisp, Python, Ruby, Perl, R, C und Mathematica.
Alles diese Sprachen haben ihre Daseinsberechtigung, jede von ihnen erfüllt einen anderen Anwendungsfall. Doch keine war richtig geeignet für wissenschaftliches Rechnen, maschinelles Lernen, Data-Mining und paralleles Rechen zu vereinigen. Mit Julia sollte eine Sprache geschaffen werden, die so schnell wie C, dynamisch wie Ruby und über bekannte mathematische Notationen wie Matlab verfügt. So fanden sich, Jeff Bezanson, Stefan Karpinski, Viral B. Shah und Alan Edelman zusammen und begannen mit der Arbeit an Julia.

Die Arbeit an Julia begann 2009, aber erst 2012 veröffentlichten die Entwickler auf einer Webseite den ersten Blog-Post und erklärten ihre Vision von Julia. 
Die Programmiersprache sollte ihre Nutzer unterstützen und ist weiterhin Open Source.
\cite{Bezanson_Julia_A_fresh_2017}

\subsection{Namensursprung}
Es ist nicht ganz klar, woher der Name Julia kommt, die wohl beste Herleitung führt
einer der Hauptentwickler, Jeff Bezanson, in einem Forumseintrag auf Jeff’s uncommon lisp is automated hin.
Was soviel heißt, wie Jeffs untypische Lisp Automatisierung und spielt damit auf seine Erweiterung der Lisp Programmiersprache an, in der er Probleme mit der Typisierung von Lisp hatte.\footnote{\url{https://discourse.julialang.org/t/cas-benchmarks-symbolics-jl-and-maxima/58359/16} Zugriff am 21.12.23}

\subsection{Community um Julia}
Seit 2014 verfügt die Community  über so viele Enthusiasten, dass man begann eine jährliche Konferenz zu veranstalten. \cite{juliacon2014}
Die Gemeinschaft wuchs und wächst von Jahr zu Jahr. In den Jahren 2020 und 2021 fand die Julia Con aufgrund von Covid-19 virtuell statt \cite{juliacon2020} und verbuchte Teilnehmerrekorde.

Mit dem stetigen Wachstum der Julia-Community und den erfolgreichen virtuellen Ausgaben der Julia Con gewann die Programmiersprache Julia zunehmend an Bedeutung. Doch wie sieht ihre heutige Relevanz und Anwendung in verschiedenen Bereichen aus?

\section{Heutige Bedeutung}
Julia wird heute in verschiedenen wissenschaftlichen Disziplinen eingesetzt,
darunter Physik, Biologie, Finanzwesen und Ingenieurwissenschaften. Die Sprache hat sich zu einem leistungsfähigen Werkzeug für Datenanalysten, Forscher und Ingenieure entwickelt, die komplexe Berechnungen durchführen müssen. \newline

Die Sprache bietet viele verschiedenen Pakete, um maschinelles Lernen zu erleichtern, Bild Verarbeitung, biologische Berechnung, Statistik, Visualisierung, Manipulieren von Daten, ein effizientes Speichern, Ein- und Ausgabe von Daten. 
\cite{roesch2023julia}


\subsection{Julia in der Biologie}
In der Biologie kann Julia helfen, die Konstruktion von Modellen zu vereinfachen, z.B. Abläufe in Zellkernen zu modellieren.
Hierbei kann das Konzept der Metaprogrammierung zum Einsatz kommen.  
Die Metaprogrammierung hat ihre Wurzeln in der LISP-Programmiersprache und ermöglicht eine Art von Reflexion und Lernen durch die Software. In Julia wird dies durch Makros umgesetzt. Diese flexiblen Codevorlagen können zur effizienten Generierung wiederkehrender Codes verwendet werden.
Julia ermöglicht es, verschiedene Modelle automatisch aus einem einzigen Codeblock zu generieren, was Arbeitsabläufe vereinfacht und effizienter macht. \cite{roesch2023julia}

\subsection{Julia in der Physik}
In der Raumfahrtbehörde NASA wird die Programmiersprache Julia zur Modellierung der Raumfahrzeugtrennung im Weltraum eingesetzt. Eine detaillierte Darstellung dieses Anwendungsgebiets wurde in einem Vortrag von Jonathan Diegelman zu Beginn des Jahres 2021 präsentiert. \footnote{\url{https://www.youtube.com/watch?v=tQpqsmwlfY0} Zugriff am 02.01.24} \newline


Im Jahr 2021 fand am Conseil Européen pour la Recherche Nucléaire (CERN) ein Workshop statt, der sich eingehend mit der Anwendung der Julia-Programmiersprache in wissenschaftlichen Projekten im Bereich der Hochenergiephysik befasste. Das CERN, als führende europäische Organisation für Kernforschung mit Sitz in Genf, Schweiz, verfolgt vorrangig das Ziel, die Grundlagen der Physik zu erforschen, insbesondere im Bereich der Teilchenphysik. \newline

Die Veranstaltung wurde mit dem ausdrücklichen Ziel durchgeführt, die Potenziale von Julia in der Hochenergiephysik zu erkunden. Der Workshop bot einen umfassenden Überblick über die Anwendungsgebiete von Julia in wissenschaftlichen Projekten. Eine detaillierte Zusammenfassung der Veranstaltung und ihrer Inhalte kann unter dem folgenden Link eingesehen werden: Diese Initiative des CERN spiegelt das Bestreben wider, die Julia-Programmiersprache gezielt in wissenschaftlichen Forschungsprojekten einzusetzen und ihre Anwendbarkeit im spezifischen Kontext der Hochenergiephysik zu begutachten.\cite{HEP_Mini_workshop}

\newpage
\subsection{Julia in der Wirtschaft}

BlackRock, der weltweit größte Vermögensverwalter mit einem verwalteten Vermögen von fast 5 Billionen US-Dollar, setzt auf Julia für seine nächste Generation von Analyseplattformen. 
Das Unternehmen hat Analytics-Module für sein Hauptprodukt Aladdin in Julia geschrieben und verwendet Julia für die Analyse von Zeitreihendaten sowie für Big-Data-Anwendungen.\cite{juliaconBlackrock} \newline

Die kanadische Zentralbank verwendet Julia für Makroökonomisches Modellierung.
Dies ist ein Prozess der Erstellung und Analyse von Modellen, die die gesamtwirtschaftliche Aktivität einer Volkswirtschaft darstellen. Diese Modelle helfen Ökonomen, Politikern und anderen Interessengruppen, die Auswirkungen von wirtschaftlichen Veränderungen, politischen Maßnahmen und anderen Faktoren auf die Gesamtwirtschaft zu verstehen. \footnote{\url{https://github.com/bankofcanada?language=julia} Zugriff am 02.01.24}


% Hauptteil 600 - 800 Wörter
% 761 Wörter

\section{Zusammenfassung}
Zusammenfassend kann man sagen, die Julia-Programmiersprache  zeichnet sich aus durch ihre Geschwindigkeit, Vielseitigkeit und der eleganten Syntax. Sie erregte die Aufmerksamkeit von Entwicklern und Forschern weltweit. Die Entstehungsgeschichte, von der Vision ihrer Gründer bis zur Gründung einer wachsenden Community, zeigt den Erfolg dieses Open-Source-Projekts. Heute findet Julia Anwendung in verschiedenen wissenschaftlichen Disziplinen wie Biologie, Physik und Finanzwesen. Unternehmen wie BlackRock setzen auf Julia für fortschrittliche Analyseplattformen, und auch die kanadische Zentralbank nutzt die Sprache für makroökonomische Modellierung. Viele der oben genannten Beispiele unterstreichen die aktuelle Relevanz und breite Anwendung von Julia.
% Schluss 5-10\% -> 50-100 Wörtern
% 91 Wörter