\usepackage[utf8]{inputenc}
\usepackage{amsfonts}
\usepackage{amssymb }
\usepackage[ngerman,english]{babel}
\usepackage{ upgreek }
\usepackage{bibgerm}
%\usepackage[square,sort,comma,numbers]{natbib}% bibliography style for support for urls in the 
\bibliographystyle{gerplain}

\usepackage[official]{eurosym}
\usepackage{graphicx}
\usepackage{caption}
\usepackage{subcaption}
\usepackage{graphviz}
\usepackage[hidelinks]{hyperref}
\usepackage[dvipsnames]{xcolor}

\usepackage{comment}
\usepackage{geometry}
\usepackage{tikz}
\geometry{a4paper, portrait,left=2.5cm, right=2.5cm, top=2cm, bottom=3.5cm}
\usepackage[T1]{fontenc}
\usepackage{lmodern}

\usepackage{array}
\usepackage{listings} % für code 

\usepackage{csquotes}
\usepackage[backend=biber, bibstyle=apa, citestyle=apa]{biblatex} % wichtig für zitieren
\addbibresource{literatur.bib}

% ---- aussehen von links definiert
\usepackage{hyperref}
\hypersetup{
    colorlinks=true,
    linkcolor=black,
    filecolor=magenta,      
    urlcolor=blue,
    citecolor=black,
}


\urlstyle{same}

\usepackage[font=itshape]{quoting}

% Für Headlines und Footer
\usepackage{fancyhdr}

\pagestyle{fancy}
\fancyhf{}
\renewcommand{\footrulewidth}{1pt}
\rhead{Belegarbeit\includegraphics[width=0.05\textwidth]{logo.png}} % Beschreibung für rechts oben
\lhead{\leftmark} % Beschreibung links oben
\rfoot{Seite \thepage} % Beschreibung rechts unten
\lfoot{Einführung in das wissenschaftliche Schreiben \\Technische Hochschule Brandenburg } % Beschreibung links unten


